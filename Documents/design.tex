\documentclass[14pt]{article}

\usepackage[T2A]{fontenc}
\usepackage[utf8]{inputenc}
\usepackage[english,russian]{babel}

\usepackage[simplified]{pgf-umlcd}
\usepackage{tikz}
\usetikzlibrary{shapes.geometric,backgrounds,calc,positioning,arrows,
                chains,decorations.markings,patterns,fadings,shapes.multipart,trees}

\begin{document}

\begin{tikzpicture}
  \begin{class}[text width=6cm]{User}{0,0}
    \attribute{id : Integer}
    \attribute{зарегистрирован : Date}
    \attribute{Фамилия : String}
    \attribute{Имя : String}
    \attribute{Отчество : String}
  \end{class}

  \begin{class}{Group}{10,0}
    \attribute{id : Integer}
    \attribute{Название : String}
    \attribute{Тип группы}
  \end{class}
  
  \begin{class}{Period}{10,5}
    \attribute{id : Integer}
    \attribute{Название : String}
    \attribute{Начало : Date}
    \attribute{Окончание : Date}
  \end{class}

  \association{User}{}{0..*}{Group}{}{0..*}
  \association{Period}{}{1}{Group}{}{0..*}

\end{tikzpicture}

\section{Сущности}

\subsection{Пользователь}

\subsection{Группа}

Группы объединяют пользователей. Каждый из них может входить в любое количество групп.

Группы могут быть двух типов: учебные и отряды в лагере.

\end{document}
