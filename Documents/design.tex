\documentclass[14pt]{article}

\usepackage[letterpaper, landscape, margin=1.5cm]{geometry}

\usepackage[T2A]{fontenc}
\usepackage[utf8]{inputenc}
\usepackage[english,russian]{babel}

\usepackage[simplified]{pgf-umlcd}
\usepackage{tikz}
\usetikzlibrary{shapes.geometric,backgrounds,calc,positioning,arrows,
                chains,decorations.markings,patterns,fadings,shapes.multipart,trees}

\begin{document}

\section{Структура базы данных}

\begin{center}
  \ttfamily
  \begin{tikzpicture}
    \begin{class}[text width=6cm]{User}{0,4}
      \attribute{id : Integer (\textbf{PK})}
      \attribute{registration : Date}
      \attribute{surname : String}
      \attribute{name : String}
      \attribute{patronymic : String}
      \attribute{role : UserRole enum}
    \end{class}
    
    \begin{class}{UserRole}{-7, 4}
    	\attribute{* Student}
    	\attribute{* Parent}
    	\attribute{* Сommittee}
    	\attribute{* Teacher}
    	\attribute{* Administrator}
    \end{class}
    
    \begin{class}{Relationship}{-7, 0}
       \attribute{id : Integer (\textbf{PK})}
       \attribute{parent : Integer (\textbf{FK})}
       \attribute{child : Integer (\textbf{FK})}
       \attribute{state : RequestState enum}
    \end{class}
    
    \begin{class}{RequestState}{-7, -4}
      \attribute{* Waiting}
      \attribute{* Approved}
      \attribute{* Declined}
    \end{class}

    \begin{class}{GroupList}{0,0}
      \attribute{id : Integer (\textbf{PK})}
      \attribute{userId : Integer (\textbf{FK})}
      \attribute{groupId : Integer (\textbf{FK})}
      \attribute{state : RequestState}
    \end{class}

    \begin{class}{Group}{7,0}
      \attribute{id : Integer (\textbf{PK})}
      \attribute{periodId : Integer (\textbf{FK})}
      \attribute{name : String}
      \attribute{isOpen : bool}
      \attribute{type : GroupType enum}
    \end{class}
  
    \begin{class}{Period}{7,4}
      \attribute{id : Integer (\textbf{PK})}
      \attribute{name : String}
      \attribute{beginDate : Date}
      \attribute{endDate : Date}
    \end{class}
    
    \begin{class}{GroupType}{7, -4}
      \attribute{* Study}
      \attribute{* Camp}
    \end{class}

    \association{User}{}{0..*}{GroupList}{}{}
    \association{User}{}{}{UserRole}{}{}
    \association{User}{}{}{Relationship}{}{}
    \association{Relationship}{}{}{RequestState}{}{}
    \association{GroupList}{}{}{Group}{}{0..*}
    \association{GroupList}{}{}{RequestState}{}{}
    \association{Period}{}{1}{Group}{}{0..*}
    \association{Group}{}{}{GroupType}{}{}
  \end{tikzpicture}
\end{center}
\section{Сущности}

\subsection{Пользователь (User)}

Пользователь сайта. 

Атрибуты:

\begin{itemize}
	\item \emph{id}~---~идентификатор пользователя, является PK;
	\item \emph{registration}~---~дата регистрации на сайте;
	\item \emph{surname}~---~Фамилия пользователя;
	\item \emph{name}~---~Имя пользователя;
	\item \emph{patronymic}~---~Отчество пользователя;
	\item \emph{role}~---~Роль пользователя в системе, является экземпляром перечисления ``Роли пользователей''.
\end{itemize}

\subsection{Роли пользователей (UserRole)}

У пользователей сайта могут быть разные роли: возможности и уровень доступа у разных пользователей могут отличаться.
Текущие роли:
\begin{itemize}
	\item Ученик (Student);
	\item Родитель (Parent);
	\item Представитель комитета (Committee);
	\item Преподаватель (Teacher);
	\item Админстратор (Administrator).
\end{itemize}

\subsection{Родственники (Relationship)}

Таблица, устанавливающая связи между пользователями с ролью ``Родитель'' и ``Ученик'' и отслеживающая состояние заявок на родство между пользователеми.

Атрибуты:
\begin{itemize}
	\item \emph{id}~---~идентификатор пользователя, является PK;
	\item \emph{parent}~---~ссылка на id пользователя-родителя, является FK;
	\item \emph{child}~---~ссылка на id пользователя-ученика (ребенка), является FK;
	\item \emph{state}~---~состояние заявки на родство между пользователями, является экземпляром перечисления ``Состояние заявки''.
\end{itemize}

\subsection{Состояние заявки (RequestState)}

Состояние заявки.

Текущие состояния:
\begin{itemize}
	\item \emph{Waiting}~---~Заявка отправлена и ожидает принятия или отказа;
	\item \emph{Approved}~---~Заявка принята;
	\item \emph{Declined}~---~В заявке отказано.
\end{itemize}

\subsection{Период обучения (Period)}

Учебный период в КШ.

Атрибуты:
\begin{itemize}
	\item \emph{id}~---~идентификатор пользователя, является PK;
	\item \emph{name}~---~название учебного периода;
	\item \emph{beginDate}~---~начало периода;
	\item \emph{endDate}~---~окончание периода.
\end{itemize}

\subsection{Группа (Group)}

Группы объединяют пользователей. Каждый из них может входить в любое количество групп.

Атрибуты:
\begin{itemize}
	\item \emph{id}~---~идентификатор пользователя, является PK;
	\item \emph{periodId}~---~ссылка на учебный период, к которому относится группа, является FK;
	\item \emph{name}~---~название учебной группы;
	\item \emph{isOpen}~---~открыт ли набор в данную группу, \emph{true} в случае, если открыт, в противном~---~\emph{false};
	\item \emph{type}~---~тип группы, является экземпляром перечисления ``Тип группы''.
\end{itemize}

\subsection{Список группы (GroupList)}

Список участников групп. Каждая запись указывается принадлежность пользователя к группе.

Атрибуты:
\begin{enumerate}
	\item \emph{id}~---~идентификатор пользователя, является PK; 
	\item \emph{userId}~---~ссылка на пользователя-участника, является FK;
	\item \emph{groupId}~---~ссылка на группу, является FK;
	\item \emph{state}~---~состояние заявки на вступление, является экземпляром перечисления ``Состояние заявки''.
\end{enumerate}


\subsection{Тип группы (GroupType)}

Группы могут быть двух типов: учебные (Study) и отряды в лагере (Camp).

\end{document}
